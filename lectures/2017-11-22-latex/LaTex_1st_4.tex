\documentclass{article}

\title{First Time LaTex}

\begin{document}

The well known Pythagorean theorem \(x^2 + y^2 = z^2\) was proved to be invalid for other exponents. Meaning the next equation has no integer solutions:   \[ x^n + y^n = z^n \]

The well known Pythagorean theorem \(x^2 + y^2 = z^2\) was proved to be invalid for other exponents. Meaning the next equation has no integer solutions:   \( x^n + y^n = z^n \) 

In physics, the mass-energy equivalence is stated by the equation $E=mc^2$, discovered in 1905 by Albert Einstein.

The mass-energy equivalence is described by the famous equation   $$E=mc^2$$   discovered in 1905 by Albert Einstein. In natural units ($c$ = 1), the formula expresses the identity   \begin{equation} E=m \end{equation}


$x^2$\\
$x^i+2$\\
$x^{i+2}$\\
\\
$x_3$\\
$\sqrt{x+2}$\\
$\frac{y+2}{5}$\\
\\
$\infty$\\
$\to$\\
$\geq$\\
$\leq$\\
$\neq$\\
$\cdot$\\
$\pi$\\
$x^3+15=33$

\[x^3+15=33\]


$$
\sum_{k=0}^\infty\frac{(-1)^k}{k+1} = \int_0^1\frac{dx}{1+x}
$$

$$
\lim_{x\rightarrow 0} \frac{\sin x}{x} = 1
$$

$\sqrt{\frac{a}{b}}$ \\\\


$\sqrt[10]{\frac{a}{b}}$

$$
\int_0^{2\pi}\cos(mx)\,dx = 0 \hspace{1cm}
\mbox{if and only if} \hspace{1cm} m\ne 0
$$


\end{document}

