%\documentclass{article}

\documentclass[titlepage]{article}

\title{First Time LaTex}

\author{Hyeong-Seok Lim}

%\pagestyle{plain}

%\pagestyle{empty}

%\pagestyle{headings}

\pagestyle{myheadings}

\begin{document}
\maketitle
\tableofcontents


\newpage

\section{Introduction}

Tamoxifen is one of the most commonly used endocrine therapeutic agents for breast cancer. One of the main active metabolites of tamoxifen, endoxifen, is transformed mainly via the cytochrome P450 2D6 (CYP2D6) enzyme. Many studies have examined whether the treatment outcomes of tamoxifen for breast cancer differ by \textit{CYP2D6} genotypes. However, previous study results in this regard have been inconsistent, and the role of \textit{CYP2D6} in the prediction of patient outcomes from tamoxifen therapy remains controversial. We therefore performed a meta-analysis of 10 previous clinical reports on this issue (n = 5,183) to evaluate the association between \textit{CYP2D6} genotypes and hazard ratios for the recurrence risk of breast cancer after postoperative tamoxifen treatment. Pooled estimates of hazard ratios were computed using R and $NONMEM^®$ software. We found from this investigation that there is a significantly increased risk of breast cancer recurrence in patients carrying variant \textit{CYP2D6} genotypes. The mean hazard ratios and 95\% confidence intervals were 1.60 [1.04–2.47] in the random effect model implemented in R and 1.63 [1.01–2.62] in the random effect model in $\textrm{NONMEM}^®$. The bootstrap result (2,000 replicates) of $\textrm{NONMEM}^®$ was 1.64 [1.07–2.79]. Our present findings thus suggest that genetic polymorphisms of \textit{CYP2D6} may be important predictors of the clinical outcomes of adjuvant tamoxifen treatment for the patients with breast cancer. A future large-scale, randomized, well-controlled trial is warranted to confirm our findings.

\newpage

\section{Method}
\label{sec:method}

This is method section.


$x^2$\\
$x^i+2$\\
$x^{i+2}$\\
\\
$x_3$\\
$\sqrt{x+2}$\\
$\frac{y+2}{5}$\\
\\
$\infty$\\
$\to$\\
$\geq$\\
$\leq$\\
$\neq$\\
$\cdot$\\
$\pi$\\
$x^3+15=33$
\[x^3+15=33\]




\newpage
\section{Result}

This is result section.

For method, please see the section \ref{sec:method}

\newpage
\section{Discussion}

This is Discussion section.

\newpage
\section{Conclusion}

This is Conclusion section.

\end{document}

